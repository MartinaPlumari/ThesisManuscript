As modern video games transition toward increasingly dense and interactive environments, traditional Object-Oriented Programming (OOP) patterns in game engines face significant CPU bottlenecks. This thesis investigates the implementation of a high-performance NPC horde system using **Unreal Engine’s Mass Framework**, a data-oriented suite leveraging **Entity Component System (ECS)** architecture. The research aims to achieve a scalable simulation of thousands of NPCs on mid-range hardware, focusing on several key technical challenges: seamless navigation across non-planar landscapes, dynamic interaction with level objects, and a hybrid movement system capable of transitioning from global path-following to local target acquisition.

The implementation utilizes **MassEntity** for memory-efficient data management and explores the integration of diverse animation states to maintain visual fidelity at scale. Preliminary results indicate that while the framework successfully handles the logic for thousands of entities, out-of-the-box features, in particular **spatial avoidance** and high-fidelity representation, introduce significant performance overhead, currently stabilizing at approximately 800 fully-simulated agents on target hardware.

Beyond technical execution, this study provides a critical analysis of Mass as a production-ready tool. The findings suggest that while the core ECS architecture offers immense potential for custom optimization, the “experimental” status of many Mass modules and the high barrier to entry for non-technical staff (such as Game Designers) necessitate a cautious approach for commercial adoption. The thesis concludes by proposing future optimizations, including custom avoidance algorithms, to further bridge the gap between experimental data-oriented technology and stable production workflows.